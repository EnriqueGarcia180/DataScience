% Options for packages loaded elsewhere
\PassOptionsToPackage{unicode}{hyperref}
\PassOptionsToPackage{hyphens}{url}
\documentclass[
]{article}
\usepackage{xcolor}
\usepackage[margin=1in]{geometry}
\usepackage{amsmath,amssymb}
\setcounter{secnumdepth}{-\maxdimen} % remove section numbering
\usepackage{iftex}
\ifPDFTeX
  \usepackage[T1]{fontenc}
  \usepackage[utf8]{inputenc}
  \usepackage{textcomp} % provide euro and other symbols
\else % if luatex or xetex
  \usepackage{unicode-math} % this also loads fontspec
  \defaultfontfeatures{Scale=MatchLowercase}
  \defaultfontfeatures[\rmfamily]{Ligatures=TeX,Scale=1}
\fi
\usepackage{lmodern}
\ifPDFTeX\else
  % xetex/luatex font selection
\fi
% Use upquote if available, for straight quotes in verbatim environments
\IfFileExists{upquote.sty}{\usepackage{upquote}}{}
\IfFileExists{microtype.sty}{% use microtype if available
  \usepackage[]{microtype}
  \UseMicrotypeSet[protrusion]{basicmath} % disable protrusion for tt fonts
}{}
\makeatletter
\@ifundefined{KOMAClassName}{% if non-KOMA class
  \IfFileExists{parskip.sty}{%
    \usepackage{parskip}
  }{% else
    \setlength{\parindent}{0pt}
    \setlength{\parskip}{6pt plus 2pt minus 1pt}}
}{% if KOMA class
  \KOMAoptions{parskip=half}}
\makeatother
\usepackage{color}
\usepackage{fancyvrb}
\newcommand{\VerbBar}{|}
\newcommand{\VERB}{\Verb[commandchars=\\\{\}]}
\DefineVerbatimEnvironment{Highlighting}{Verbatim}{commandchars=\\\{\}}
% Add ',fontsize=\small' for more characters per line
\usepackage{framed}
\definecolor{shadecolor}{RGB}{248,248,248}
\newenvironment{Shaded}{\begin{snugshade}}{\end{snugshade}}
\newcommand{\AlertTok}[1]{\textcolor[rgb]{0.94,0.16,0.16}{#1}}
\newcommand{\AnnotationTok}[1]{\textcolor[rgb]{0.56,0.35,0.01}{\textbf{\textit{#1}}}}
\newcommand{\AttributeTok}[1]{\textcolor[rgb]{0.13,0.29,0.53}{#1}}
\newcommand{\BaseNTok}[1]{\textcolor[rgb]{0.00,0.00,0.81}{#1}}
\newcommand{\BuiltInTok}[1]{#1}
\newcommand{\CharTok}[1]{\textcolor[rgb]{0.31,0.60,0.02}{#1}}
\newcommand{\CommentTok}[1]{\textcolor[rgb]{0.56,0.35,0.01}{\textit{#1}}}
\newcommand{\CommentVarTok}[1]{\textcolor[rgb]{0.56,0.35,0.01}{\textbf{\textit{#1}}}}
\newcommand{\ConstantTok}[1]{\textcolor[rgb]{0.56,0.35,0.01}{#1}}
\newcommand{\ControlFlowTok}[1]{\textcolor[rgb]{0.13,0.29,0.53}{\textbf{#1}}}
\newcommand{\DataTypeTok}[1]{\textcolor[rgb]{0.13,0.29,0.53}{#1}}
\newcommand{\DecValTok}[1]{\textcolor[rgb]{0.00,0.00,0.81}{#1}}
\newcommand{\DocumentationTok}[1]{\textcolor[rgb]{0.56,0.35,0.01}{\textbf{\textit{#1}}}}
\newcommand{\ErrorTok}[1]{\textcolor[rgb]{0.64,0.00,0.00}{\textbf{#1}}}
\newcommand{\ExtensionTok}[1]{#1}
\newcommand{\FloatTok}[1]{\textcolor[rgb]{0.00,0.00,0.81}{#1}}
\newcommand{\FunctionTok}[1]{\textcolor[rgb]{0.13,0.29,0.53}{\textbf{#1}}}
\newcommand{\ImportTok}[1]{#1}
\newcommand{\InformationTok}[1]{\textcolor[rgb]{0.56,0.35,0.01}{\textbf{\textit{#1}}}}
\newcommand{\KeywordTok}[1]{\textcolor[rgb]{0.13,0.29,0.53}{\textbf{#1}}}
\newcommand{\NormalTok}[1]{#1}
\newcommand{\OperatorTok}[1]{\textcolor[rgb]{0.81,0.36,0.00}{\textbf{#1}}}
\newcommand{\OtherTok}[1]{\textcolor[rgb]{0.56,0.35,0.01}{#1}}
\newcommand{\PreprocessorTok}[1]{\textcolor[rgb]{0.56,0.35,0.01}{\textit{#1}}}
\newcommand{\RegionMarkerTok}[1]{#1}
\newcommand{\SpecialCharTok}[1]{\textcolor[rgb]{0.81,0.36,0.00}{\textbf{#1}}}
\newcommand{\SpecialStringTok}[1]{\textcolor[rgb]{0.31,0.60,0.02}{#1}}
\newcommand{\StringTok}[1]{\textcolor[rgb]{0.31,0.60,0.02}{#1}}
\newcommand{\VariableTok}[1]{\textcolor[rgb]{0.00,0.00,0.00}{#1}}
\newcommand{\VerbatimStringTok}[1]{\textcolor[rgb]{0.31,0.60,0.02}{#1}}
\newcommand{\WarningTok}[1]{\textcolor[rgb]{0.56,0.35,0.01}{\textbf{\textit{#1}}}}
\usepackage{graphicx}
\makeatletter
\newsavebox\pandoc@box
\newcommand*\pandocbounded[1]{% scales image to fit in text height/width
  \sbox\pandoc@box{#1}%
  \Gscale@div\@tempa{\textheight}{\dimexpr\ht\pandoc@box+\dp\pandoc@box\relax}%
  \Gscale@div\@tempb{\linewidth}{\wd\pandoc@box}%
  \ifdim\@tempb\p@<\@tempa\p@\let\@tempa\@tempb\fi% select the smaller of both
  \ifdim\@tempa\p@<\p@\scalebox{\@tempa}{\usebox\pandoc@box}%
  \else\usebox{\pandoc@box}%
  \fi%
}
% Set default figure placement to htbp
\def\fps@figure{htbp}
\makeatother
\setlength{\emergencystretch}{3em} % prevent overfull lines
\providecommand{\tightlist}{%
  \setlength{\itemsep}{0pt}\setlength{\parskip}{0pt}}
\usepackage{bookmark}
\IfFileExists{xurl.sty}{\usepackage{xurl}}{} % add URL line breaks if available
\urlstyle{same}
\hypersetup{
  pdftitle={Regresión Lineal Simple},
  pdfauthor={David Nexticapan Cortes},
  hidelinks,
  pdfcreator={LaTeX via pandoc}}

\title{Regresión Lineal Simple}
\author{David Nexticapan Cortes}
\date{2025-03-08}

\begin{document}
\maketitle

Cargamos librearías

\begin{Shaded}
\begin{Highlighting}[]
\FunctionTok{library}\NormalTok{(readxl)}
\FunctionTok{library}\NormalTok{(ggplot2)}
\FunctionTok{library}\NormalTok{(dplyr)}
\FunctionTok{library}\NormalTok{(gridExtra) }\CommentTok{\# Forma de organizar los gráficos.}
\FunctionTok{library}\NormalTok{(GGally)}
\FunctionTok{library}\NormalTok{(corrplot)}
\FunctionTok{library}\NormalTok{(car) }\CommentTok{\# qqPlot de los residuales.}
\FunctionTok{library}\NormalTok{(lmtest) }\CommentTok{\# Prueba de autocorrelación de DW y prueba de Breusch{-}Pagan}
\FunctionTok{library}\NormalTok{(nortest) }\CommentTok{\# Prueba de KS}
\FunctionTok{library}\NormalTok{(MASS)}
\FunctionTok{library}\NormalTok{(caTools) }\CommentTok{\# sample.split}
\FunctionTok{library}\NormalTok{(tseries) }\CommentTok{\#Prueba de Jarque Bera}
\end{Highlighting}
\end{Shaded}

Cargamos la base de datos

\begin{Shaded}
\begin{Highlighting}[]
\NormalTok{Pulso }\OtherTok{\textless{}{-}} \FunctionTok{read\_excel}\NormalTok{(}\StringTok{"PulsoCardiacoMujeres.xlsx"}\NormalTok{)}
\FunctionTok{head}\NormalTok{(Pulso)}
\end{Highlighting}
\end{Shaded}

\begin{verbatim}
## # A tibble: 6 x 5
##    Edad Estatura  Peso Cintura Pulso
##   <dbl>    <dbl> <dbl>   <dbl> <dbl>
## 1    17     1.63  51.7    67.2    76
## 2    32     1.69  67.2    82.5    72
## 3    25     1.58  48.5    66.7    88
## 4    55     1.58  72      93      60
## 5    27     1.51  57.2    82.6    72
## 6    29     1.62  55.4    75.4    68
\end{verbatim}

\textbf{Análisis Exploratorio}

Verificamos si tenemos la presencia de valores NA

\begin{Shaded}
\begin{Highlighting}[]
\FunctionTok{colSums}\NormalTok{(}\FunctionTok{is.na}\NormalTok{(Pulso)) }\CommentTok{\# No hay presencia de valores NA}
\end{Highlighting}
\end{Shaded}

\begin{verbatim}
##     Edad Estatura     Peso  Cintura    Pulso 
##        0        0        0        0        0
\end{verbatim}

Realizamos un resumen de estadísticas descriptivas de la variable de
interés.

\begin{Shaded}
\begin{Highlighting}[]
\FunctionTok{summary}\NormalTok{(Pulso}\SpecialCharTok{$}\NormalTok{Cintura)}
\end{Highlighting}
\end{Shaded}

\begin{verbatim}
##    Min. 1st Qu.  Median    Mean 3rd Qu.    Max. 
##   66.70   73.42   81.95   85.03   94.38  126.50
\end{verbatim}

\begin{itemize}
\item
  La mediana del contorno de la cintura es menor que la media, lo que
  sugiere una asimetría a la derecha.
\item
  El 25\% de las mujeres de la muestra tienen una cintura mayor que
  94.38
\item
  Existe una gran diferencia del contorno de la cintura entre el valor
  mínimo y el valor máximo.
\end{itemize}

Realizamos un boxplot y análisis de ouliers

\begin{Shaded}
\begin{Highlighting}[]
\FunctionTok{boxplot}\NormalTok{(Pulso}\SpecialCharTok{$}\NormalTok{Cintura) }\CommentTok{\# No hay presencia de outliers en la variable de interés.}
\end{Highlighting}
\end{Shaded}

\pandocbounded{\includegraphics[keepaspectratio]{10---Regresión-Lineal_files/figure-latex/unnamed-chunk-5-1.pdf}}

\begin{Shaded}
\begin{Highlighting}[]
\FunctionTok{boxplot.stats}\NormalTok{(Pulso}\SpecialCharTok{$}\NormalTok{Cintura)}
\end{Highlighting}
\end{Shaded}

\begin{verbatim}
## $stats
## [1]  66.70  73.25  81.95  94.75 126.50
## 
## $n
## [1] 40
## 
## $conf
## [1] 76.57887 87.32113
## 
## $out
## numeric(0)
\end{verbatim}

\begin{itemize}
\item
  Tenemos el valor del bigote inferior = 66.70 que coincide con el valor
  mínimo de los datos.
\item
  Tenemos el valor del bigote superior = 126.50 que coincide con el
  valor máximo de los datos.
\item
  No hay presencia de outliers en los datos.
\end{itemize}

Matriz de correlaciones

\begin{Shaded}
\begin{Highlighting}[]
\NormalTok{matriz\_corr }\OtherTok{\textless{}{-}} \FunctionTok{cor}\NormalTok{(}\FunctionTok{select\_if}\NormalTok{(Pulso, is.numeric)) }\CommentTok{\# Para filtrar del data Set solo las variables Numericas. En este caso no tiene pero en caso de que tuviera. podria ser: matriz\_corr \textless{}{-} cor(Pulso) y en este caso daria o mismo}
\NormalTok{matriz\_corr}
\end{Highlighting}
\end{Shaded}

\begin{verbatim}
##               Edad  Estatura      Peso   Cintura     Pulso
## Edad     1.0000000 0.1167019 0.5630050 0.6594466 0.3079243
## Estatura 0.1167019 1.0000000 0.3610981 0.1887012 0.2079803
## Peso     0.5630050 0.3610981 1.0000000 0.9210571 0.1434836
## Cintura  0.6594466 0.1887012 0.9210571 1.0000000 0.1327250
## Pulso    0.3079243 0.2079803 0.1434836 0.1327250 1.0000000
\end{verbatim}

Diagrama de calor para ver la matriz de correlaciones.

\begin{Shaded}
\begin{Highlighting}[]
\FunctionTok{corrplot}\NormalTok{(matriz\_corr, }\AttributeTok{method =} \StringTok{"color"}\NormalTok{, }\AttributeTok{addCoef.col =} \StringTok{"black"}\NormalTok{)}
\end{Highlighting}
\end{Shaded}

\pandocbounded{\includegraphics[keepaspectratio]{10---Regresión-Lineal_files/figure-latex/unnamed-chunk-7-1.pdf}}

\begin{itemize}
\item
  Tenemos una correlación positiva muy fuerte entre la cintura y el peso
  de una mujer.
\item
  Tenemos una correlación positiva moderada entre las variables Peso -
  Edad y Cintura - Edad.
\item
  Tenemos una correlación positiva debil entre las demás variables.
\item
  No tenemos relaciones negativas o inversas.
\end{itemize}

Matriz de gráficos de dispersión

\begin{Shaded}
\begin{Highlighting}[]
\FunctionTok{pairs}\NormalTok{(Pulso, }\AttributeTok{main =} \StringTok{"Matriz de gráficos de dispersión"}\NormalTok{, }\AttributeTok{pch =} \DecValTok{20}\NormalTok{, }\AttributeTok{col =}  \StringTok{"blue"}\NormalTok{, }\AttributeTok{lwd =} \DecValTok{3}\NormalTok{)}
\end{Highlighting}
\end{Shaded}

\pandocbounded{\includegraphics[keepaspectratio]{10---Regresión-Lineal_files/figure-latex/unnamed-chunk-8-1.pdf}}

\begin{Shaded}
\begin{Highlighting}[]
\CommentTok{\# Es clara la correlación positiva entre el peso y la cintura de una mujer.}
\CommentTok{\# pch es la forma de los puntos}
\end{Highlighting}
\end{Shaded}

\begin{itemize}
\tightlist
\item
  Se observa una relación directamente lineal entre la cintura y el
  peso. Es la mas evidente.
\end{itemize}

Histogramas para cada variable

\begin{Shaded}
\begin{Highlighting}[]
\NormalTok{variables }\OtherTok{\textless{}{-}} \FunctionTok{c}\NormalTok{(}\StringTok{"Edad"}\NormalTok{, }\StringTok{"Estatura"}\NormalTok{, }\StringTok{"Peso"}\NormalTok{, }\StringTok{"Cintura"}\NormalTok{, }\StringTok{"Pulso"}\NormalTok{)}

\CommentTok{\# Configura una ventana de gráficos (ajusta según el número de variables)}
\FunctionTok{par}\NormalTok{(}\AttributeTok{mfrow =} \FunctionTok{c}\NormalTok{(}\DecValTok{3}\NormalTok{, }\DecValTok{2}\NormalTok{))  }\CommentTok{\# 3 filas, 2 columnas (ajustable)}

\CommentTok{\# Recorremos cada variable para graficarla}
\ControlFlowTok{for}\NormalTok{ (var }\ControlFlowTok{in}\NormalTok{ variables) \{}
\NormalTok{  datos }\OtherTok{\textless{}{-}}\NormalTok{ Pulso[[var]] }\CommentTok{\# dos corchetes para tener un arreglo de numeros. Un corchete saca un data frame con los numeros. Necesitamos arreglo de datos para el histograma.}
  \FunctionTok{hist}\NormalTok{(datos, }\AttributeTok{probability =} \ConstantTok{TRUE}\NormalTok{, }\AttributeTok{main =} \FunctionTok{paste}\NormalTok{(}\StringTok{"Distribución de"}\NormalTok{, var), }\AttributeTok{xlab =}\NormalTok{ var, }\AttributeTok{col =} \StringTok{"salmon1"}\NormalTok{, }\AttributeTok{breaks =} \FunctionTok{nclass.Sturges}\NormalTok{(datos))}
  \FunctionTok{lines}\NormalTok{(}\FunctionTok{density}\NormalTok{(datos), }\AttributeTok{col =} \StringTok{"black"}\NormalTok{, }\AttributeTok{lwd =} \DecValTok{3}\NormalTok{) }\CommentTok{\# EN COLOR NEGRO linea EMPIRICA}
  \FunctionTok{curve}\NormalTok{(}\FunctionTok{dnorm}\NormalTok{(x, }\AttributeTok{mean =} \FunctionTok{mean}\NormalTok{(datos), }\AttributeTok{sd =} \FunctionTok{sd}\NormalTok{(datos)), }\AttributeTok{add =} \ConstantTok{TRUE}\NormalTok{, }\AttributeTok{col =} \StringTok{"blue"}\NormalTok{, }\AttributeTok{lwd =} \DecValTok{3}\NormalTok{) }\CommentTok{\# EN COLOR AZUL linea DISTRIBUCION NORMAL}
\NormalTok{\}}
\CommentTok{\# paste: Pegar un texto con un numero}
\end{Highlighting}
\end{Shaded}

\pandocbounded{\includegraphics[keepaspectratio]{10---Regresión-Lineal_files/figure-latex/unnamed-chunk-9-1.pdf}}

\textbf{Aplicación del Modelo}

Seleccionamos como nuestra variable de respuesta a ``Peso'' (``y'') y
nuestra variable predictora a ``Cintura'' (``x'')

Dividir nuestros datos en datos de entrenamiento y datos de prueba. 80\%
de los datos para Entrenar 20\% de los datos para Probar

\begin{Shaded}
\begin{Highlighting}[]
\FunctionTok{set.seed}\NormalTok{(}\DecValTok{42}\NormalTok{)}

\CommentTok{\# Barajeamos la base de datos, en caso de que la base de datos venga ordenada por algun criterio.}
\NormalTok{barajeo }\OtherTok{\textless{}{-}} \FunctionTok{slice\_sample}\NormalTok{(Pulso, }\AttributeTok{prop =} \DecValTok{1}\NormalTok{) }\CommentTok{\# bareajear una proporcion del 100\% o sea a todas.}

\CommentTok{\# Dividimos aleatoriamente los datos en proporcion 80{-}20}
\NormalTok{split }\OtherTok{\textless{}{-}} \FunctionTok{sample.split}\NormalTok{(barajeo}\SpecialCharTok{$}\NormalTok{Edad, }\AttributeTok{SplitRatio =} \FloatTok{0.8}\NormalTok{) }\CommentTok{\# Vector lógico }

\CommentTok{\# Base de entrenamiento}
\NormalTok{train }\OtherTok{\textless{}{-}} \FunctionTok{subset}\NormalTok{(barajeo, split }\SpecialCharTok{==} \ConstantTok{TRUE}\NormalTok{)}

\CommentTok{\# Base de Prueba o Testeo}
\NormalTok{test }\OtherTok{\textless{}{-}} \FunctionTok{subset}\NormalTok{(barajeo, split }\SpecialCharTok{==} \ConstantTok{FALSE}\NormalTok{)}
\end{Highlighting}
\end{Shaded}

Ajustamos el modelo con los datos de entrenamiento

\begin{Shaded}
\begin{Highlighting}[]
\NormalTok{modelo }\OtherTok{\textless{}{-}} \FunctionTok{lm}\NormalTok{(Peso }\SpecialCharTok{\textasciitilde{}}\NormalTok{ Cintura, }\AttributeTok{data =}\NormalTok{ train)}
\FunctionTok{summary}\NormalTok{(modelo)}
\end{Highlighting}
\end{Shaded}

\begin{verbatim}
## 
## Call:
## lm(formula = Peso ~ Cintura, data = train)
## 
## Residuals:
##      Min       1Q   Median       3Q      Max 
## -12.5478  -5.1178   0.2585   3.6588  17.7889 
## 
## Coefficients:
##              Estimate Std. Error t value Pr(>|t|)    
## (Intercept) -17.50199    7.08877  -2.469   0.0195 *  
## Cintura       0.98806    0.07996  12.358 2.67e-13 ***
## ---
## Signif. codes:  0 '***' 0.001 '**' 0.01 '*' 0.05 '.' 0.1 ' ' 1
## 
## Residual standard error: 7.145 on 30 degrees of freedom
## Multiple R-squared:  0.8358, Adjusted R-squared:  0.8303 
## F-statistic: 152.7 on 1 and 30 DF,  p-value: 2.669e-13
\end{verbatim}

El modelo ajustado es:

\[
\hat{Peso} = -17.50199 + 0.98806 *Cintura
\]

\textbf{Elementos del Modelo:}

test Fisher:

En regresión simple, el test F se utiliza para probar si el modelo con
la variable explicativa es significativamente mejor que un modelo que
solo incluye la media. El juego de hipótesis que se plantea es:

\[
H_0: \beta_1 = 0 \\
H_1: \beta_1 \neq 0
\] - En el test de Fisher tenemos un p-value: 2.669e-13 \textless{} 0.05
se rechaza la hipótesis nula en favor de la alternativa.

\begin{itemize}
\tightlist
\item
  Existe suficiente evidencia estadística para sustentar la aseveración
  de que al menos una beta es diferente de cero.
\end{itemize}

Coeficientes:

\begin{itemize}
\item
  El p-valor de la variable ``Cintura'' y el intercepto son menores que
  el nivel de significancia, indicando que son significativas para el
  modelo.
\item
  Intersección (Intercept) = -17.50199: Cuando la variable Cintura = 0,
  el modelo predice que el peso sería aproximadamente -17.50199 kg. Esta
  interpretación no tiene sentido práctico, ya que una cintura de 0 cm
  no es realista. Sin embargo, el intercepto es necesario para calcular
  los valores ajustados.
\item
  Pendiente (Cintura) = 0.98806: Por cada aumento de 1 cm en la cintura,
  el peso promedio aumenta en 0.98806 kg.
\end{itemize}

Residuales:

\begin{itemize}
\item
  La mediana de los residuos 0.2585 es cercana a 0, lo cual sugiere que
  el modelo no tiene un sesgo fuerte en la predicción ya que el error
  típico está cercano a cero.
\item
  Aunque el modelo se ajusta bastante bien, hay algunos errores
  (residuos) un poco grandes.
\end{itemize}

R-squared

\begin{itemize}
\tightlist
\item
  El modelo explica el 83.58\% de la variabilidad en el peso usando la
  cintura como predictor. Esto indica un ajuste fuerte.
\end{itemize}

\begin{Shaded}
\begin{Highlighting}[]
\CommentTok{\# Gráfico de dispersión con línea de regresión}
\FunctionTok{plot}\NormalTok{(train}\SpecialCharTok{$}\NormalTok{Cintura, train}\SpecialCharTok{$}\NormalTok{Peso, }\AttributeTok{col=}\StringTok{\textquotesingle{}blue3\textquotesingle{}}\NormalTok{, }\AttributeTok{pch =} \DecValTok{16}\NormalTok{, }\AttributeTok{xlab =} \StringTok{"Cintura"}\NormalTok{, }\AttributeTok{ylab =} \StringTok{"Peso"}\NormalTok{, }\AttributeTok{main =} \StringTok{"Regresión Lineal: Peso \textasciitilde{} Cintura"}\NormalTok{, }\AttributeTok{lwd =} \DecValTok{5}\NormalTok{)}
\FunctionTok{abline}\NormalTok{(modelo, }\AttributeTok{col=}\StringTok{\textquotesingle{}red\textquotesingle{}}\NormalTok{, }\AttributeTok{lwd =} \DecValTok{2}\NormalTok{)}
\end{Highlighting}
\end{Shaded}

\pandocbounded{\includegraphics[keepaspectratio]{10---Regresión-Lineal_files/figure-latex/unnamed-chunk-12-1.pdf}}

\textbf{Verificamos supuestos de e}

Pruebas de Normalidad

\begin{Shaded}
\begin{Highlighting}[]
\CommentTok{\# Histograma de los residuales}
\FunctionTok{hist}\NormalTok{(}\FunctionTok{residuals}\NormalTok{(modelo), }\AttributeTok{probability =}\NormalTok{ T,}\AttributeTok{col=}\StringTok{\textquotesingle{}salmon\textquotesingle{}}\NormalTok{, }\AttributeTok{main=}\StringTok{\textquotesingle{}Histograma de residuos\textquotesingle{}}\NormalTok{, }\AttributeTok{breaks =} \FunctionTok{nclass.Sturges}\NormalTok{(modelo}\SpecialCharTok{$}\NormalTok{residuals))}
\FunctionTok{lines}\NormalTok{(}\FunctionTok{density}\NormalTok{(}\FunctionTok{residuals}\NormalTok{(modelo)), }\AttributeTok{col =} \StringTok{"black"}\NormalTok{, }\AttributeTok{lwd =} \DecValTok{3}\NormalTok{)}
\FunctionTok{curve}\NormalTok{(}\FunctionTok{dnorm}\NormalTok{(x, }\FunctionTok{mean}\NormalTok{(}\FunctionTok{residuals}\NormalTok{(modelo)), }\FunctionTok{sd}\NormalTok{(}\FunctionTok{residuals}\NormalTok{(modelo))), }\AttributeTok{col =} \StringTok{"red3"}\NormalTok{, }\AttributeTok{lwd =} \DecValTok{3}\NormalTok{, }\AttributeTok{add =}\NormalTok{ T)}
\end{Highlighting}
\end{Shaded}

\pandocbounded{\includegraphics[keepaspectratio]{10---Regresión-Lineal_files/figure-latex/unnamed-chunk-13-1.pdf}}

\begin{Shaded}
\begin{Highlighting}[]
\CommentTok{\# QQ{-}NORM}
\FunctionTok{qqPlot}\NormalTok{(}\FunctionTok{residuals}\NormalTok{(modelo), }\AttributeTok{pch =} \DecValTok{20}\NormalTok{, }\AttributeTok{col =} \StringTok{"black"}\NormalTok{, }\AttributeTok{lwd =} \DecValTok{2}\NormalTok{)}
\end{Highlighting}
\end{Shaded}

\pandocbounded{\includegraphics[keepaspectratio]{10---Regresión-Lineal_files/figure-latex/unnamed-chunk-14-1.pdf}}

\begin{verbatim}
## [1] 13  8
\end{verbatim}

\begin{Shaded}
\begin{Highlighting}[]
\CommentTok{\# test de normalidad}
\FunctionTok{shapiro.test}\NormalTok{(}\FunctionTok{residuals}\NormalTok{(modelo)) }\CommentTok{\# Si hay normalidad en los residuales.}
\end{Highlighting}
\end{Shaded}

\begin{verbatim}
## 
##  Shapiro-Wilk normality test
## 
## data:  residuals(modelo)
## W = 0.97392, p-value = 0.6137
\end{verbatim}

\begin{itemize}
\item
  Como p-value = 0.6137 \textgreater{} 0.05 no se rechaza la hipótesis
  nula.
\item
  Existe suficiente evidencia estadística para sustentar la aseveración
  de que los residuales siguen una distribución significativamente
  normal.
\end{itemize}

Homocedasticida (Igualdad de varianzas): La igualdad de varianzas se
espera entre los residuos y los valores ajustados del modelo.

\begin{Shaded}
\begin{Highlighting}[]
\FunctionTok{plot}\NormalTok{(}\FunctionTok{fitted}\NormalTok{(modelo), }\FunctionTok{residuals}\NormalTok{(modelo), }\AttributeTok{col =} \StringTok{\textquotesingle{}blue\textquotesingle{}}\NormalTok{, }\AttributeTok{pch =} \DecValTok{16}\NormalTok{, }\AttributeTok{main =} \StringTok{\textquotesingle{}Fitted vs Residuals\textquotesingle{}}\NormalTok{)}
\FunctionTok{abline}\NormalTok{(}\AttributeTok{h=}\DecValTok{0}\NormalTok{, }\AttributeTok{col=}\StringTok{\textquotesingle{}red\textquotesingle{}}\NormalTok{)}
\end{Highlighting}
\end{Shaded}

\pandocbounded{\includegraphics[keepaspectratio]{10---Regresión-Lineal_files/figure-latex/unnamed-chunk-16-1.pdf}}

\begin{Shaded}
\begin{Highlighting}[]
\CommentTok{\# fitted(modelo): Son los valores ajustados o predichos por tu modelo. }
\CommentTok{\# residuals(modelo): Son los residuos.}
\end{Highlighting}
\end{Shaded}

\begin{itemize}
\tightlist
\item
  Como los puntos están dispersos de forma aleatoria arriba y abajo de
  la línea roja y no hay patrones se sugiere que si hay
  homocedasticidad.
\end{itemize}

La Prueba de \textbf{Breusch-Pagan} es una prueba estadística que se
utiliza para detectar la heterocedasticidad en un modelo de regresión
lineal. La heterocedasticidad ocurre cuando la varianza de los errores o
residuos no es constante a lo largo de las observaciones, lo que
violaría una de las suposiciones básicas de los modelos de regresión
lineal (homocedasticidad).

\(H_0:\) La varianza de los errores es constante (homocedasticidad)
\(H_1:\) La varianza de los errores no es constante (heterocedasticidad)

\begin{Shaded}
\begin{Highlighting}[]
\FunctionTok{bptest}\NormalTok{(modelo)}
\end{Highlighting}
\end{Shaded}

\begin{verbatim}
## 
##  studentized Breusch-Pagan test
## 
## data:  modelo
## BP = 1.8447, df = 1, p-value = 0.1744
\end{verbatim}

\begin{itemize}
\item
  Como p-value = 0.1744 \textgreater{} 0.05 no se rechaza la hipótesis
  nula.
\item
  Existe suficiente evidencia estadística para sustentar la aseveración
  de que la varianza de los errores es constante.
\end{itemize}

El test de \textbf{Jarque Bera} es una prueba de normalidad que evalúa
si una muestra de datos (en este caso, los residuos del modelo) tiene
una asimetría (skewness) y una curtosis (kurtosis) cercanas a las de una
distribución normal.

\(H_0:\) Los residuos siguen una distribución normal (sin sesgo ni
curtosis excesiva) \(H_1:\) Los residuos no siguen una distribución
normal (pueden tener sesgo y curtosis fuera de los parámetros normales)

\begin{Shaded}
\begin{Highlighting}[]
\FunctionTok{jarque.bera.test}\NormalTok{(}\FunctionTok{residuals}\NormalTok{(modelo))}
\end{Highlighting}
\end{Shaded}

\begin{verbatim}
## 
##  Jarque Bera Test
## 
## data:  residuals(modelo)
## X-squared = 0.85534, df = 2, p-value = 0.652
\end{verbatim}

\begin{itemize}
\item
  Como p-value = 0.652 \textgreater{} 0.05 no se rechaza la hipótesis
  nula,
\item
  Existe suficiente evidencia estadística para sustentar la aseveración
  de que los residuos siguen una distribución normal.
\end{itemize}

Uno de los supuestos del modelo de regresión lineal es que los residuos
deben ser independientes entre sí.

La prueba de \textbf{Durbin-Watson} evalúa si existe autocorrelación de
primer orden en los residuos del modelo de regresión, es decir, si los
errores están correlacionados entre sí (especialmente entre un error y
el anterior).

El estadístico de Durbin Watson toma valores entre 0 y 4:

\begin{itemize}
\item
  DW ≈ 2 → No hay autocorrelación
\item
  DW \textless{} 2 → Autocorrelación positiva
\item
  DW \textgreater{} 2 → Autocorrelación negativa
\end{itemize}

\begin{Shaded}
\begin{Highlighting}[]
\CommentTok{\# Prueba de Durbin{-}Watson para autocorrelación de residuos}
\CommentTok{\# H\_0: No hay autocorrelación en los residuos (es decir, los residuos son independientes entre sí).}
\CommentTok{\# H\_1: Sí hay autocorrelación en los residuos}
\FunctionTok{dwtest}\NormalTok{(modelo)}
\end{Highlighting}
\end{Shaded}

\begin{verbatim}
## 
##  Durbin-Watson test
## 
## data:  modelo
## DW = 2.2082, p-value = 0.7528
## alternative hypothesis: true autocorrelation is greater than 0
\end{verbatim}

\begin{itemize}
\tightlist
\item
  Como DW = 2.2082 lo cual está muy cerca de 2, entonces no hay
  evidencia de autocorrelación significativa. Esto indica que los
  errores del modelo son independientes.
\end{itemize}

\textbf{Predicciones}

\begin{Shaded}
\begin{Highlighting}[]
\CommentTok{\# Predicciones en el conjunto de prueba}
\NormalTok{predicciones }\OtherTok{\textless{}{-}} \FunctionTok{predict}\NormalTok{(modelo, }\AttributeTok{newdata =}\NormalTok{ test, }\AttributeTok{interval =} \StringTok{"confidence"}\NormalTok{, }\AttributeTok{level =} \FloatTok{0.95}\NormalTok{)}

\CommentTok{\# Guardar las predicciones y los intervalos de confianza en el conjunto de prueba}
\NormalTok{test}\SpecialCharTok{$}\NormalTok{predicted }\OtherTok{\textless{}{-}}\NormalTok{ predicciones[,}\DecValTok{1}\NormalTok{]}
\NormalTok{test}\SpecialCharTok{$}\NormalTok{lower }\OtherTok{\textless{}{-}}\NormalTok{ predicciones[,}\DecValTok{2}\NormalTok{]}
\NormalTok{test}\SpecialCharTok{$}\NormalTok{upper }\OtherTok{\textless{}{-}}\NormalTok{ predicciones[,}\DecValTok{3}\NormalTok{]}

\CommentTok{\# Ver el conjunto de prueba con las nuevas columnas de predicciones}
\NormalTok{test}
\end{Highlighting}
\end{Shaded}

\begin{verbatim}
## # A tibble: 8 x 8
##    Edad Estatura  Peso Cintura Pulso predicted lower upper
##   <dbl>    <dbl> <dbl>   <dbl> <dbl>     <dbl> <dbl> <dbl>
## 1    17     1.63  51.7    67.2    76      48.9  44.7  53.1
## 2    25     1.52  50.3    73.6    80      55.2  51.8  58.6
## 3    55     1.58  72      93      60      74.4  71.6  77.1
## 4    27     1.51  57.2    82.6    72      64.1  61.4  66.8
## 5    40     1.49  42.4    67.8    80      49.5  45.4  53.6
## 6    22     1.54  55.9    75.5    64      57.1  53.9  60.3
## 7    29     1.62  55.4    75.4    68      57.0  53.8  60.2
## 8    23     1.64  49      74.5    72      56.1  52.8  59.4
\end{verbatim}

\textbf{Valores Influyentes}

Columnas del output:

dfb.1\_: Este valor muestra el cambio en el coeficiente estimado para la
variable independiente ``Cintura'' si se elimina la observación en
cuestión. Si es muy grande (por ejemplo, mayor que 1), puede indicar que
esa observación tiene un efecto fuerte sobre el modelo.

dfb.Cntr: Similar al anterior, pero para el intercepto. Nos dice cómo
cambia el intercepto del modelo si eliminamos la observación
correspondiente.

dffit: Este valor es una medida de cuán ``lejana'' está una observación
de la predicción ajustada para su valor. Un valor alto de dffit indica
que esa observación puede estar alejándose de la relación general del
modelo. Valores mayores a 2 o 3 suelen ser motivo de preocupación.

cov.r: Esta columna mide el cambio en la covarianza del modelo cuando se
elimina una observación. Un valor cercano a 1 sugiere que la observación
tiene un fuerte impacto en la covarianza.

cook.d: El estadístico de Cook (cook.d) es una medida que evalúa la
influencia de una observación en el modelo de regresión en general.
Básicamente, mide cuánto cambiarían los resultados del modelo si esa
observación fuera eliminada. Si el valor de Cook es grande (generalmente
mayor que 1), indica que esa observación tiene un gran impacto en los
resultados del modelo y podría ser considerada influyente.

hat: La palanca (hat) mide cuán influyente es una observación en la
predicción de su propio valor dentro del modelo. Si una observación
tiene una palanca alta, significa que un pequeño cambio en esa
observación podría afectar mucho su propia predicción, lo que indica que
tiene un impacto considerable en cómo el modelo la interpreta. Esto
también puede influir indirectamente en las predicciones de otras
observaciones, ya que cambia la forma en que el modelo ajusta su
estructura general.

\begin{Shaded}
\begin{Highlighting}[]
\CommentTok{\# Evaluamos los valores influyentes}
\NormalTok{influencia }\OtherTok{\textless{}{-}} \FunctionTok{influence.measures}\NormalTok{(modelo)}

\CommentTok{\# Mostramos las medidas de influencia}
\FunctionTok{summary}\NormalTok{(influencia)}
\end{Highlighting}
\end{Shaded}

\begin{verbatim}
## Potentially influential observations of
##   lm(formula = Peso ~ Cintura, data = train) :
## 
##    dfb.1_ dfb.Cntr dffit cov.r   cook.d hat    
## 8   0.11  -0.04     0.41  0.80_*  0.07   0.03  
## 11 -0.02   0.02     0.03  1.37_*  0.00   0.22_*
## 13 -0.30   0.40     0.65  0.69_*  0.17   0.05  
## 26 -0.56   0.62     0.66  1.24_*  0.22   0.22_*
\end{verbatim}

\begin{Shaded}
\begin{Highlighting}[]
\CommentTok{\# Para identificar puntos con alta influencia, se puede filtrar:}
\NormalTok{influencia}\SpecialCharTok{$}\NormalTok{summaries[}\FunctionTok{which}\NormalTok{(influencia}\SpecialCharTok{$}\NormalTok{summaries}\SpecialCharTok{$}\NormalTok{cook.d }\SpecialCharTok{\textgreater{}} \DecValTok{1}\NormalTok{),]}
\end{Highlighting}
\end{Shaded}

\begin{verbatim}
## NULL
\end{verbatim}

Observación 8:

\begin{itemize}
\item
  cook.d = 0.80: Esto es un valor elevado. El estadístico de Cook mayor
  que 1 indica que la observación podría tener un gran impacto en la
  estimación del modelo.
\item
  hat = 0.03: Esto indica que la observación no tiene una gran
  influencia en la predicción de su propio valor.
\end{itemize}

Observación 11:

cook.d = 1.37: Este es un valor significativamente mayor que 1, lo que
sugiere que esta observación tiene un fuerte impacto en el modelo.

hat = 0.22: Este valor de ``hat'' es relativamente alto, lo que indica
que esta observación tiene una gran influencia sobre las predicciones.

Observación 13:

\begin{itemize}
\item
  cook.d = 0.17: Este valor es bajo, lo que sugiere que esta observación
  no tiene una gran influencia en el modelo.
\item
  hat = 0.05: Este valor también es bajo, lo que indica que la
  observación no tiene una gran influencia sobre la predicción de su
  propio valor.
\end{itemize}

Observación 26:

\begin{itemize}
\item
  cook.d = 1.24: Este valor es elevado, lo que sugiere que esta
  observación tiene una gran influencia en el modelo.
\item
  hat = 0.22: Similar a la observación 11, esta observación tiene una
  alta influencia sobre las predicciones.
\end{itemize}

\begin{Shaded}
\begin{Highlighting}[]
\CommentTok{\# Graficar influencia}
\FunctionTok{plot}\NormalTok{(modelo, }\AttributeTok{which =} \DecValTok{4}\SpecialCharTok{:}\DecValTok{5}\NormalTok{)}
\end{Highlighting}
\end{Shaded}

\pandocbounded{\includegraphics[keepaspectratio]{10---Regresión-Lineal_files/figure-latex/unnamed-chunk-22-1.pdf}}
\pandocbounded{\includegraphics[keepaspectratio]{10---Regresión-Lineal_files/figure-latex/unnamed-chunk-22-2.pdf}}

\begin{itemize}
\item
  La mayoría de los puntos están concentrados en el centro.
\item
  El punto 26 tiene un leverage alto lo que puede estar influenciando
  mucho en la forma del modelo.
\item
  El punto 2 y 13 tienen un residuo grande aunque con leverage bajo.
  Podríamos sugerir que el modelo no los predijo bien.
\end{itemize}

\begin{Shaded}
\begin{Highlighting}[]
\NormalTok{train[}\FunctionTok{c}\NormalTok{(}\DecValTok{2}\NormalTok{,}\DecValTok{13}\NormalTok{,}\DecValTok{26}\NormalTok{),]}
\end{Highlighting}
\end{Shaded}

\begin{verbatim}
## # A tibble: 3 x 5
##    Edad Estatura  Peso Cintura Pulso
##   <dbl>    <dbl> <dbl>   <dbl> <dbl>
## 1    59     1.61  73.4   105.     76
## 2    41     1.72  98.5    99.4    68
## 3    57     1.61 115.    126.     80
\end{verbatim}

\#\textbf{Modelo 2}

Probamos el modelo sin la observación 26

\begin{Shaded}
\begin{Highlighting}[]
\NormalTok{train2 }\OtherTok{\textless{}{-}}\NormalTok{ train[}\SpecialCharTok{{-}}\DecValTok{26}\NormalTok{,]}
\end{Highlighting}
\end{Shaded}

Ajustamos el modelo con los datos de entrenamiento

\begin{Shaded}
\begin{Highlighting}[]
\NormalTok{modelo2 }\OtherTok{\textless{}{-}} \FunctionTok{lm}\NormalTok{(Peso }\SpecialCharTok{\textasciitilde{}}\NormalTok{ Cintura, }\AttributeTok{data =}\NormalTok{ train2)}
\FunctionTok{summary}\NormalTok{(modelo2)}
\end{Highlighting}
\end{Shaded}

\begin{verbatim}
## 
## Call:
## lm(formula = Peso ~ Cintura, data = train2)
## 
## Residuals:
##      Min       1Q   Median       3Q      Max 
## -11.3836  -4.8204  -0.4341   3.4696  18.6940 
## 
## Coefficients:
##              Estimate Std. Error t value Pr(>|t|)    
## (Intercept) -13.54823    7.72056  -1.755   0.0899 .  
## Cintura       0.93918    0.08857  10.603 1.72e-11 ***
## ---
## Signif. codes:  0 '***' 0.001 '**' 0.01 '*' 0.05 '.' 0.1 ' ' 1
## 
## Residual standard error: 7.083 on 29 degrees of freedom
## Multiple R-squared:  0.795,  Adjusted R-squared:  0.7879 
## F-statistic: 112.4 on 1 and 29 DF,  p-value: 1.718e-11
\end{verbatim}

El modelo ajustado es: \[
\hat{Peso} = -13.54823 + 0.93918 *Cintura
\]

\pandocbounded{\includegraphics[keepaspectratio]{Puntos Influyentes.jpg}}

\begin{itemize}
\item
  La observación 26 era influyente ya que su presencia hacía que el
  intercepto fuera más bajo.
\item
  El coeficiente de cintura cambió moderadamente, pero relevante si usas
  el modelo para predicciones.
\item
  La observación 26 ayudaba a que el modelo explicara el 83.56\% de la
  variabilidad y sin él baja al 79.5\%.
\item
  En conclusión, la observación 26 si es influyente ya que al quitarla
  el modelo cambia pero no se destruye y el ajuste sigue siendo bueno
  pero un poco menos explicativo.
\item
  En el test de Fisher tenemos un p-value: 2.669e-13 \textless{} 0.05 se
  rechaza la hipótesis nula en favor de la alternativa. Es decir,
  tenemos al menos una beta diferente de cero.
\end{itemize}

\begin{Shaded}
\begin{Highlighting}[]
\CommentTok{\# Gráfico de dispersión con línea de regresión}
\FunctionTok{plot}\NormalTok{(train2}\SpecialCharTok{$}\NormalTok{Cintura, train2}\SpecialCharTok{$}\NormalTok{Peso, }\AttributeTok{col=}\StringTok{\textquotesingle{}blue3\textquotesingle{}}\NormalTok{, }\AttributeTok{pch =} \DecValTok{16}\NormalTok{, }\AttributeTok{xlab =} \StringTok{"Cintura"}\NormalTok{, }\AttributeTok{ylab =} \StringTok{"Peso"}\NormalTok{, }\AttributeTok{main =} \StringTok{"Regresión Lineal: Peso \textasciitilde{} Cintura"}\NormalTok{, }\AttributeTok{lwd =} \DecValTok{5}\NormalTok{)}
\FunctionTok{abline}\NormalTok{(modelo2, }\AttributeTok{col=}\StringTok{\textquotesingle{}red\textquotesingle{}}\NormalTok{, }\AttributeTok{lwd =} \DecValTok{2}\NormalTok{)}
\end{Highlighting}
\end{Shaded}

\pandocbounded{\includegraphics[keepaspectratio]{10---Regresión-Lineal_files/figure-latex/unnamed-chunk-26-1.pdf}}

\textbf{Verificamos supuestos de e}

Pruebas de Normalidad

\begin{Shaded}
\begin{Highlighting}[]
\CommentTok{\# Histograma de los residuales}
\FunctionTok{hist}\NormalTok{(}\FunctionTok{residuals}\NormalTok{(modelo2), }\AttributeTok{probability =}\NormalTok{ T,}\AttributeTok{col=}\StringTok{\textquotesingle{}salmon\textquotesingle{}}\NormalTok{, }\AttributeTok{main=}\StringTok{\textquotesingle{}Histograma de residuos\textquotesingle{}}\NormalTok{, }\AttributeTok{breaks =} \FunctionTok{nclass.Sturges}\NormalTok{(modelo}\SpecialCharTok{$}\NormalTok{residuals))}
\FunctionTok{lines}\NormalTok{(}\FunctionTok{density}\NormalTok{(}\FunctionTok{residuals}\NormalTok{(modelo2)), }\AttributeTok{col =} \StringTok{"black"}\NormalTok{, }\AttributeTok{lwd =} \DecValTok{3}\NormalTok{)}
\FunctionTok{curve}\NormalTok{(}\FunctionTok{dnorm}\NormalTok{(x, }\FunctionTok{mean}\NormalTok{(}\FunctionTok{residuals}\NormalTok{(modelo2)), }\FunctionTok{sd}\NormalTok{(}\FunctionTok{residuals}\NormalTok{(modelo2))), }\AttributeTok{col =} \StringTok{"red3"}\NormalTok{, }\AttributeTok{lwd =} \DecValTok{3}\NormalTok{, }\AttributeTok{add =}\NormalTok{ T)}
\end{Highlighting}
\end{Shaded}

\pandocbounded{\includegraphics[keepaspectratio]{10---Regresión-Lineal_files/figure-latex/unnamed-chunk-27-1.pdf}}

\begin{Shaded}
\begin{Highlighting}[]
\CommentTok{\# QQ{-}NORM}
\FunctionTok{qqPlot}\NormalTok{(}\FunctionTok{residuals}\NormalTok{(modelo2), }\AttributeTok{pch =} \DecValTok{20}\NormalTok{, }\AttributeTok{col =} \StringTok{"black"}\NormalTok{, }\AttributeTok{lwd =} \DecValTok{2}\NormalTok{)}
\end{Highlighting}
\end{Shaded}

\pandocbounded{\includegraphics[keepaspectratio]{10---Regresión-Lineal_files/figure-latex/unnamed-chunk-28-1.pdf}}

\begin{verbatim}
## [1] 13  8
\end{verbatim}

\begin{Shaded}
\begin{Highlighting}[]
\CommentTok{\# test de normalidad}
\FunctionTok{shapiro.test}\NormalTok{(}\FunctionTok{residuals}\NormalTok{(modelo2)) }\CommentTok{\# Si hay normalidad en los residuales.}
\end{Highlighting}
\end{Shaded}

\begin{verbatim}
## 
##  Shapiro-Wilk normality test
## 
## data:  residuals(modelo2)
## W = 0.96041, p-value = 0.2991
\end{verbatim}

\begin{itemize}
\item
  Como p-value = 0.2991 \textgreater{} 0.05 no se rechaza la hipótesis
  nula.
\item
  Existe suficiente evidencia estadística para sustentar la aseveración
  de que los residuales siguen una distribución significativamente
  normal.
\end{itemize}

Homocedasticida (Igualdad de varianzas): La igualdad de varianzas se
espera entre los residuos y los valores ajustados del modelo.

\begin{Shaded}
\begin{Highlighting}[]
\FunctionTok{plot}\NormalTok{(}\FunctionTok{fitted}\NormalTok{(modelo2), }\FunctionTok{residuals}\NormalTok{(modelo2), }\AttributeTok{col =} \StringTok{\textquotesingle{}blue\textquotesingle{}}\NormalTok{, }\AttributeTok{pch =} \DecValTok{16}\NormalTok{, }\AttributeTok{main =} \StringTok{\textquotesingle{}Fitted vs Residuals\textquotesingle{}}\NormalTok{)}
\FunctionTok{abline}\NormalTok{(}\AttributeTok{h=}\DecValTok{0}\NormalTok{, }\AttributeTok{col=}\StringTok{\textquotesingle{}red\textquotesingle{}}\NormalTok{)}
\end{Highlighting}
\end{Shaded}

\pandocbounded{\includegraphics[keepaspectratio]{10---Regresión-Lineal_files/figure-latex/unnamed-chunk-30-1.pdf}}

\begin{itemize}
\tightlist
\item
  Como los puntos están dispersos de forma aleatoria arriba y abajo de
  la línea roja y no hay patrones se sugiere que si hay
  homocedasticidad.
\end{itemize}

La Prueba de \textbf{Breusch-Pagan} \(H_0:\) La varianza de los errores
es constante (homocedasticidad) \(H_1:\) La varianza de los errores no
es constante (heterocedasticidad)

\begin{Shaded}
\begin{Highlighting}[]
\FunctionTok{bptest}\NormalTok{(modelo2)}
\end{Highlighting}
\end{Shaded}

\begin{verbatim}
## 
##  studentized Breusch-Pagan test
## 
## data:  modelo2
## BP = 1.719, df = 1, p-value = 0.1898
\end{verbatim}

\begin{itemize}
\item
  Como p-value = 0.1898 \textgreater{} 0.05 no se rechaza la hipótesis
  nula.
\item
  Existe suficiente evidencia estadística para sustentar la aseveración
  de que la varianza de los errores es constante.
\end{itemize}

El test de \textbf{Jarque Bera} \(H_0:\) Los residuos siguen una
distribución normal (sin sesgo ni curtosis excesiva) \(H_1:\) Los
residuos no siguen una distribución normal (pueden tener sesgo y
curtosis fuera de los parámetros normales)

\begin{Shaded}
\begin{Highlighting}[]
\FunctionTok{jarque.bera.test}\NormalTok{(}\FunctionTok{residuals}\NormalTok{(modelo2))}
\end{Highlighting}
\end{Shaded}

\begin{verbatim}
## 
##  Jarque Bera Test
## 
## data:  residuals(modelo2)
## X-squared = 2.1051, df = 2, p-value = 0.3491
\end{verbatim}

\begin{itemize}
\item
  Como p-value = 0.3491 \textgreater{} 0.05 no se rechaza la hipótesis
  nula,
\item
  Existe suficiente evidencia estadística para sustentar la aseveración
  de que los residuos siguen una distribución normal.
\end{itemize}

La prueba de \textbf{Durbin-Watson}

\begin{itemize}
\item
  DW ≈ 2 → No hay autocorrelación
\item
  DW \textless{} 2 → Autocorrelación positiva
\item
  DW \textgreater{} 2 → Autocorrelación negativa
\end{itemize}

\begin{Shaded}
\begin{Highlighting}[]
\FunctionTok{dwtest}\NormalTok{(modelo2)}
\end{Highlighting}
\end{Shaded}

\begin{verbatim}
## 
##  Durbin-Watson test
## 
## data:  modelo2
## DW = 2.1106, p-value = 0.6448
## alternative hypothesis: true autocorrelation is greater than 0
\end{verbatim}

\begin{itemize}
\tightlist
\item
  Como DW = 2.1106 lo cual está muy cerca de 2, entonces no hay
  evidencia de autocorrelación significativa. Esto indica que los
  errores del modelo son independientes.
\end{itemize}

\begin{Shaded}
\begin{Highlighting}[]
\CommentTok{\# Predicciones en el conjunto de prueba}
\NormalTok{test2 }\OtherTok{\textless{}{-}}\NormalTok{ test[,}\FunctionTok{c}\NormalTok{(}\DecValTok{1}\SpecialCharTok{:}\DecValTok{5}\NormalTok{)]}
\NormalTok{predicciones2 }\OtherTok{\textless{}{-}} \FunctionTok{predict}\NormalTok{(modelo2, }\AttributeTok{newdata =}\NormalTok{ test2, }\AttributeTok{interval =} \StringTok{"confidence"}\NormalTok{, }\AttributeTok{level =} \FloatTok{0.95}\NormalTok{)}

\CommentTok{\# Guardar las predicciones y los intervalos de confianza en el conjunto de prueba}
\NormalTok{test2}\SpecialCharTok{$}\NormalTok{predicted }\OtherTok{\textless{}{-}}\NormalTok{ predicciones2[,}\DecValTok{1}\NormalTok{]}
\NormalTok{test2}\SpecialCharTok{$}\NormalTok{lower }\OtherTok{\textless{}{-}}\NormalTok{ predicciones2[,}\DecValTok{2}\NormalTok{]}
\NormalTok{test2}\SpecialCharTok{$}\NormalTok{upper }\OtherTok{\textless{}{-}}\NormalTok{ predicciones2[,}\DecValTok{3}\NormalTok{]}

\CommentTok{\# Ver el conjunto de prueba con las nuevas columnas de predicciones}
\NormalTok{test2}
\end{Highlighting}
\end{Shaded}

\begin{verbatim}
## # A tibble: 8 x 8
##    Edad Estatura  Peso Cintura Pulso predicted lower upper
##   <dbl>    <dbl> <dbl>   <dbl> <dbl>     <dbl> <dbl> <dbl>
## 1    17     1.63  51.7    67.2    76      49.6  45.3  53.8
## 2    25     1.52  50.3    73.6    80      55.6  52.1  59.0
## 3    55     1.58  72      93      60      73.8  70.9  76.7
## 4    27     1.51  57.2    82.6    72      64.0  61.4  66.7
## 5    40     1.49  42.4    67.8    80      50.1  45.9  54.3
## 6    22     1.54  55.9    75.5    64      57.4  54.1  60.6
## 7    29     1.62  55.4    75.4    68      57.3  54.0  60.5
## 8    23     1.64  49      74.5    72      56.4  53.1  59.8
\end{verbatim}

\end{document}
